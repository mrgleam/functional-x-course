\documentclass[12pt,t]{beamer}
\usepackage{minted, amsmath, amssymb, graphicx}
\setbeameroption{hide notes}
\setbeamertemplate{note page}[plain]

\usetheme{default}
\beamertemplatenavigationsymbolsempty
\hypersetup{pdfpagemode=UseNone} % don't show bookmarks on initial view

% font
\usepackage{fontspec}
\setsansfont{Helvetica}
\setbeamerfont{note page}{family*=pplx,size=\footnotesize}

% named colors
\definecolor{offwhite}{RGB}{249,242,215}
\definecolor{foreground}{RGB}{255,255,255}
\definecolor{background}{RGB}{24,24,24}
\definecolor{title}{RGB}{107,174,214}
\definecolor{gray}{RGB}{155,155,155}
\definecolor{subtitle}{RGB}{102,255,204}
\definecolor{hilight}{RGB}{102,255,204}
\definecolor{vhilight}{RGB}{255,111,207}
\definecolor{lolight}{RGB}{155,155,155}
\definecolor{blue}{RGB}{30, 144, 255}
\definecolor{orange}{RGB}{255, 140, 0}
%\definecolor{green}{RGB}{125,250,125}

% use those colors
\setbeamercolor{titlelike}{fg=title}
\setbeamercolor{subtitle}{fg=subtitle}
\setbeamercolor{institute}{fg=gray}
\setbeamercolor{normal text}{fg=foreground,bg=background}
\setbeamercolor{item}{fg=foreground} % color of bullets
\setbeamercolor{subitem}{fg=gray}
\setbeamercolor{itemize/enumerate subbody}{fg=gray}
\setbeamertemplate{itemize subitem}{{\textendash}}
\setbeamerfont{itemize/enumerate subbody}{size=\footnotesize}
\setbeamerfont{itemize/enumerate subitem}{size=\footnotesize}

% page number
\setbeamertemplate{footline}{%
  \raisebox{5pt}{\makebox[\paperwidth]{\hfill\makebox[20pt]{\color{gray}
      \scriptsize\insertframenumber}}}\hspace*{5pt}}

% add a bit of space at the top of the notes page
\addtobeamertemplate{note page}{\setlength{\parskip}{12pt}}

% a few macros
\newcommand{\bi}{\begin{itemize}}
\newcommand{\ei}{\end{itemize}}
\newcommand{\ig}{\includegraphics}
\newcommand{\subt}[1]{{\footnotesize \color{subtitle} {#1}}}

% title into
\title{Functional X}

\begin{document}

% title slide
{
\setbeamertemplate{footline}{} % no page number here
\frame{
  \titlepage
  \note{These are slides for a talk I will give on XX XXX 202X, at ...}
}
}

\begin{frame}{Greatest Common Divisor (GCD)}
    \centering
    \Large % Increase font size
    \textcolor{orange}{\[
        \gcd : (\mathbb{N} \times \mathbb{N}) \to \mathbb{N}
    \]}
        \vspace{0.5cm} % Add spacing
    \textcolor{blue}{
        \begin{align*}
            \gcd(m,n) &= \max \big\{ d \in \mathbb{N} \mid \\
            &\quad m \bmod d = 0 \wedge n \bmod d = 0 \big\}
        \end{align*}
    }
\end{frame}

\begin{frame}{Euclid}

  \[
\textbf{euclid} \quad : \quad (\mathbb{N} \times \mathbb{N}) \to \mathbb{N}
\]

\[
\textbf{euclid}(m, n) =
\begin{cases}
    \textbf{euclid}(n, m \bmod n), & \text{if } n > 0 \\
    m, & \textbf{otherwise}
\end{cases}
\]

\end{frame}

\begin{frame}[fragile]{Function Schema}
\begin{center}
\vfill
\begin{minted}[frame=single, escapeinside=@@, bgcolor=lightgray, fontsize=\footnotesize, linenos]{haskell}
-- Haskell function schema
f :: t@$_1$@ -> ... t@$_n$@ -> t@$_r$@
f x@$_1$@ ... x@$_n$@ = if p then g else h
  where
    -- The expression p is a predicate in x@$_1$@ ...x@$_n$@
    p = x@$_1$@ > 0
    -- The expression g and h are expression involving
    -- the values x@$_1$@ ...x@$_n$@
    g = f x@$_2$@ (x@$_1$@ `mod` x@$_2$@)
    h = x@$_1$@
\end{minted}
\vfill
\end{center}
\end{frame}

\begin{frame}{Function Schema}
\vfill
\begin{center}
\begin{itemize}
  \item The symbolic name \emph{f} stands for the name of the function to be captured.
  \item The number \emph{f} gives the number of arguments of the function \emph{f}.
  \item The \emph{x$_1$ ...x$_n$} are the arguments of \emph{f}.
  \item The \emph{t$_1$ ...t$_n$} are the types of the arguments \emph{x$_1$ ...x$_n$} respectively.
  \item The type of the function result is \emph{t$_r$}.
\end{itemize}
\end{center}
\vfill
\end{frame}

\begin{frame}[fragile]{Rust Function Schema}
\begin{center}
\vfill
\begin{minted}[frame=single, escapeinside=@@, bgcolor=lightgray, fontsize=\footnotesize, linenos]{hs}
-- Rust function schema
fn f(x@$_1$@: t@$_1$@, ..., x@$_n$@: t@$_n$@) -> t@$_r$@ {
    if p {
        return g;
    } else {
        return h;
    }
}
\end{minted}
\vfill
\end{center}
\end{frame}

\end{document}
